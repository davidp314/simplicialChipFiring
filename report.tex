\documentclass[12pt]{article}

\usepackage{amsmath}
\usepackage{amssymb}

\usepackage{xcolor}

%\usepackage[textheight=8.5in, margin=1.6in]{geometry}
\usepackage[total={6in,8in}]{geometry}

%\usepackage{tikz}
%\usepackage{pgfplots}
%\tikzstyle{ball} = [circle,shading=ball, ball color=black,
%    minimum size=1mm,inner sep=1.3pt]
%\usetikzlibrary{arrows.meta}
%\tikzset{>={Latex[length=2mm,width=1.5mm]}}

%\usepackage{graphicx}

\usepackage{needspace}  % <-- to prevent bad pagebreak over environment titles
% Use:  \Needspace{3\baselineskip} before opening an environment

\usepackage{enumitem}
\setlist[enumerate]{leftmargin=*}

%\usepackage[colorlinks=true,citecolor=black,linkcolor=black,urlcolor=blue]{hyperref}

\newcommand{\N}{\mathbb{N}}
\newcommand{\Z}{\mathbb{Z}}
\newcommand{\Q}{\mathbb{Q}}
\newcommand{\R}{\mathbb{R}}
\newcommand{\C}{\mathbb{C}}
\newcommand{\T}{\mathbf{T}}
\newcommand{\sn}{\mathfrak{S}}
\newcommand{\ve}{\varepsilon}
\newcommand{\ips}[2]{\langle{#1}, {#2}\rangle}
\DeclareMathOperator{\tr}{tr}
\DeclareMathOperator{\Span}{\mathrm{Span}}
\DeclareMathOperator{\rk}{\mathrm{rank}}
\DeclareMathOperator{\trace}{\mathrm{tr}}
\DeclareMathOperator{\im}{\mathrm{im}}
\DeclareMathOperator{\Sym}{\mathrm{Sym}}
\DeclareMathOperator{\Hom}{\mathrm{Hom}}
\DeclareMathOperator{\diag}{\mathrm{diag}}
\DeclareMathOperator{\glb}{\mathrm{glb}}
\DeclareMathOperator{\lub}{\mathrm{lub}}

\parskip = 0.05in
\parindent = 0.0in

%\pagestyle{empty}

\begin{document}
\centerline{Notes on divisors on simplicial complexes.}
\bigskip

Let~$\Delta$ be a~$d$-dimensional simplicial complex, and suppose
that~$0$ is a winnable degree in dimension~$d-1$.  This means that all chains of
dimension~$d-1$ with all-zero degrees are winnable.  This is the same as saying
that
\[
  (\ker L_{d-1})^{\perp}/\im(L_{d-1})=0.
\]
However, we know that
\[
  (\ker L_{d-1})^{\perp}/\im(L_{d-1})\approx
  \T(K_{d-1}(\Delta)):=\T(\ker\partial_{d-2}/\im(L_{d-1}))
\]
where~$\T$ stands for ``torsion part''.  

Next, suppose that~$\Delta$ has a~$(d-1)$-dimensional spanning tree~$\Upsilon$, which is
the same as assuming~$\tilde{\beta}_{d-2}(\Delta)=0$, i.e.,~$\widetilde{H}_{d-2}(\Delta)$ is
torsion (QUESTION: is this true?).  Since~$\Upsilon$ is a $(d-1)$-tree, we have
\[
  \tilde{H}_{d-1}(\Upsilon)=0,\quad\text{and}\quad~\tilde{\beta}_{d-2}(\Upsilon)=0,
\]
and from this it follows that
\[
  f_{d-1}(\Upsilon)
  =f_{d-1}(\Delta)-\tilde{\beta}_{d-1}(\Delta)+\tilde{\beta}_{d-2}(\Delta)
  =f_{d-1}(\Delta)-\tilde{\beta}_{d-1}(\Delta).
\]
If we further assume that~$\widetilde{H}_{d-2}(\Upsilon)=0$, then there is an
isomorphism
\[
  K_{d-1}(\Delta)\to\Z\tilde{F}_{d-1}/\im(\tilde{L}_{d-1})
\]
where~$\tilde{F}_{d-1}$ are the~$(d-1)$-dimensional faces that are not
in~$\Upsilon$ and~$\tilde{L}_{d-1}$ is the reduced Laplacian with respect
to~$\Upsilon$.  (What happens if~$\widetilde{H}_{d-2}(\Upsilon)\neq 0$?  Is
  there still a map?  In our case, we know that~$K_{d-1}(\Delta)$ has no
  torsion.  This implies that~$\det(\tilde{L}_{d-1})=1$.

Is it surjective?)

The~$d$-th tree number for~$\Delta$ is
\[
  \tau_n(\Delta):=\sum_{\Psi}|\widetilde{H}_{d-1}(\Psi)|^2
\]
where the sum is over all~$(d-1)$-dimensional spanning trees of~$\Delta$.
By the simplicial matrix-tree theorem,
\[
  \tau_n(\Delta)
  =\frac{|\widetilde{H}_{d-2}(\Delta)|^2}{|\widetilde{H}_{d-2}(\Upsilon)|^2}\det(\tilde{L}_{d-1})
  =|\widetilde{H}_{d-2}(\Delta)|^2.
\]
(since~$\det(\tilde{L}_{d-1})=1$ and~$\widetilde{H}_{d-2}(\Upsilon)=0$).

Now, assume further that~$\widetilde{H}_{d-2}(\Delta)=0$.  It then follows
that~$\tau_n(\Delta)=1$.  It follows that~$\Delta$ has a
unique~$d$-dimensional spanning tree~$\Psi$.  Since~$\Delta$ has a
~$d$-dimensional spanning tree, it follows that $\tilde{\beta}_{d-1}(\Delta)=0$,
and hence,
\[
  f_{d}(\Psi)
  =f_{d}(\Delta)-\tilde{\beta}_{d}(\Delta)+\tilde{\beta}_{d-1}(\Delta)
  =f_{d}(\Delta)-\tilde{\beta}_{d}(\Delta).
\]
Since~$\Psi$ is a $d$-spanning tree, it has the same~$(d-1)$-skeleton. It will
equal~$\Delta$ if it has the same facets.  For sake of contradiction, suppose it
does not.  Then the above formula says that~$\tilde{\beta}_{d}(\Delta)>0$.  This
means there is a nonzero~$d$-chain in the kernel of~$\partial_d$.  This chain
contains at least two facets~$\sigma_1$ and~$\sigma_2$ in its support.
Fix~$\sigma_1$ and remove~$\sigma_2$ from~$\Delta$.  We get a new
complex~$\Delta'$.  Working over the rational numbers, we may assume 

\end{document}


